\section{Introducci\'on}
%
\subsection{Indicaciones}
%
Para el examen final deberá presentar dos archivos: el primero es un colab con todo el análisis realizado y las conclusiones pertinentes, donde el código debe poder ejecutarse. Y el segundo archivo consiste en un documento Word donde se entregará el informe final seleccionando la información que surgió como importante del análisis realizado\footnote{A pedido del alumno el formato del informe es pdf.}. En base a este informe deberá realizar una presentación para exponer el día del examen, haciendo hincapié en las cuestiones más relevantes, y proporcionando información (tablas, gráficos, medidas) que permitan apoyar la información que se presente.\\

La defensa debe tener una duración aproximada de 25 minutos. Luego de la exposición el jurado realizará preguntas que podrán ser sobre el código implementado, o sobre lo presentado y las cuestiones que validan lo realizado, o sobre la no aplicación de alguna herramienta estudiada en el cursado.\\

El enlace de descarga del dataset asignado para defender en el examen final: 
\href{https://drive.google.com/file/d/1NnOvk3UtVGYxelG_qlK_TkaYu47sYthx/view?pli=1}{Salarios.csv}.\\

El detalle de las variables se muestran en el Cuadro \ref{table:detalle_csv}.

\begin{table}[ht]
\centering
\begin{small}
\begin{tabular}{p{.3\linewidth}|p{.7\linewidth}}
Columna             & Descripción\\
\hline
work\_year          & El año en que se pagó el salario.\\
experience\_level   & El nivel de experiencia en el trabajo durante el año con los siguientes valores posibles: EN Entry-level / Junior MI Mid-level / Intermediate SE Senior-level / Expert EX Executive-level / Director\\
employment\_type    & El tipo de empleo para el puesto: PT A tiempo parcial FT A tiempo completo CT Contrato FL Freelance\\
job\_title          & El rol que desempe\~n\'o durante el año.\\
sueldo              & El monto total del salario bruto pagado.\\
salary\_currency    & La moneda del salario pagado como un código de moneda ISO 4217.\\
salarioenusd        & El salario en USD (tasa de cambio dividida por la tasa promedio de USD para el año respectivo a través de fxdata.foorilla.com).\\
employee\_residence & El país de residencia principal del empleado durante el año laboral como un código de país ISO 3166.\\
remote\_ratio       & La cantidad total de trabajo realizado de forma remota, los valores posibles son los siguientes: 0 Sin trabajo remoto (menos del 20\%) 50 Parcialmente remoto 100 Totalmente remoto (más del 80\%)\\
company\_location   & El país de la oficina principal o sucursal contratante del empleador como un código de país ISO 3166.\\
company\_size       & El número promedio de personas que trabajaron para la empresa durante el año: S menos de 50 empleados (pequeños) M 50 a 250 empleados (medianos) L más de 250 empleados (grandes)                    
\end{tabular}
\end{small}
\caption{Detalles de la tabla dada}
\label{table:detalle_csv}
\end{table}


\subsection{Objetivo}
%
El conjunto de datos asignado corresponde a salarios de trabajadores de análisis de datos, usted deberá tratar de obtener información descriptiva sobre los salarios y encontrar características de los salarios mejores pagos.\\
